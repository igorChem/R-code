\documentclass[a4paper,12pt]{article}

%===========================================================
%  packages 
\usepackage[utf8]{inputenc}
\usepackage[portuges]{babel}
\usepackage{graphics}
\usepackage{graphicx}
\usepackage{grffile}



\author{Igor Barden Grillo   \\}
\title{Modelagem Estátistica de medidas de Viscosidade e Capacidade Calorífica para soluções aquosas de Líquidos Iônicos}

\begin{document}
\date{}
\maketitle
\frenchspacing

\newpage
%===========================================================

\section{Introdução}

%===========================================================
\section{Modelagem para dados de Vicosidade}

\subsection{Estátisticas descritivas}

%===========================================================
% table with descirptive statistics for viscosity modelling 
\begin{table}[htp]\centering
\begin{tabular}{|c|c|c|p{2.5cm}|c|c|c|}
 \hline
Estatistica & Viscosidade & Densidade  & Fracao massica de LI & Volume  molar &  Temperatura \\ \hline   
Mínimo   & 0.404 & 0.9778 & 0.0000 & 141.5 & 298.0 \\ \hline                                                                                                                                                    
Média  & 15.974 & 1.1658 & 0.5749 & 209.6 & 309.6   \\ \hline      
Mediana &  7.640 & 1.2070 & 0.5525 & 221.5 & 303.0  \\ \hline      
Máximo & 161.050  & 1.3290 & 1.0000 & 291.7 & 343.0 \\ \hline      
Desvio Padrão & 26.623 & 0.1105 &  0.364 & 32.44 & 14.47  \\ \hline      

\end{tabular}
\caption{Parametros de diatribuição das variáveis utilizadas na modelagem da viscosiade}
\end{table}
%===========================================================

%===========================================================
% table with correlation  statistics for viscosity modelling 
\begin{table}[htp]\centering
\begin{tabular}{p{2.8cm}|c|c|p{2.5cm}|p{2.5cm}|c|c|c|}
 \hline
 &  Viscosidade & Densidade  & Fracao massica de LI & Volume  molar &  Temperatura \\ \hline   
Viscosidade  & 1 & 0.4258 & 0.2946 & -0.2570 &  -0.1411  \\ \hline                                                                                                                                                    
Densidade  &0.42584 &  1&0.9176  & 0.0761 & 0.3957   \\ \hline      
Fracao massica de LI &  0.2946 & 0.9176  & 1  &0.26572  &0.5513 \\ \hline      
Volume  molar & -0.1411  & 0.0761 &0.2657  & 1  &0.6331  \\ \hline      
Temperatura & -0.2570 &  & 0.5513  & 0.6331 & 1  \\ \hline      

\end{tabular}
\caption{Fator de correlação entre as variáveis}
\end{table}
%===========================================================

%===========================================================
% histogramas das variáveis 


%===========================================================
% 

\subsection{Modelos Lineares}

\section{Modelagem para dados de Capacidade Calorífica}

\subsection{Estátisticas descritivas}
\subsection{Modelos Lineares}

\section{Sumário}

\indent

\end{document}cmcm